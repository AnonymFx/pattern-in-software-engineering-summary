%!TEX root = ../report.tex

\section{Terminology}
\begin{description}
  \item[Coupling] measures the dependencies between subsystems

  \item[Cohesion] Measures the dependencies among classes within a subsystem

  \item[Design Pattern] describes associations and collaborations of a set of classes.

  \item[Architectural Style] is a pattern for a subsystem decomposition, i.e. describes relationships and collaborations of different subsystems.

  \item[Software Architecture] is an instance of an architectural style.

  \item[User Model] is imagined by the user in their mind.
  It helps the user to know and understand the underlaying application domain model.

  \item[Natural Mapping (UI)] is a mapping between UI controls of a system and objects in the real world such that the mapping does not tax the user's memory when performing a task that involves the manipulation of these controls.

  \item[Components/Subsystems] Computational units with a specified interface

  \item[Connectors/Communication] Interactions between the components/subsystems

  \item[Failure] Deviation of the observed behavior from the specified one

  \item[Fault/Bug] Mechanical or algorithmic cause of an error

  \item[Error] The system is in a state such that further processing by the system can lead to a failure.

  \item[Verification] Activity that checks if the observed behavior complies with the specified behavior of the system

  \item[Validation] Activity that checks if the observed behavior meets the needs informally expressed by a stakeholder
  
  \item[Marshalling] Transforming object to common representation and serializing afterwards to send over network
  
  \item[Unmarshalling] Deserializing data from network and transform the created object to a representation understandable by the receiver
  
\end{description}
