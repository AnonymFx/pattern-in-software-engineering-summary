%!tex root = ../report.tex

\section{Global Software Engineering Patterns}
\textbf{Diseconomy of Scale:}
The more members working on a project, the more communication is needed so that the programmer productivity decreases.\\
\textbf{Conway's Law:}
Communication structures determine how the product will look like.
Organize development effort should be organized according to the need of communication.


\paragraph{Metrics of Communication}
\begin{itemize}[topsep=5pt, itemsep=0pt]
  \item Communication and Awareness
  \item Team Collaboration
  \item Management and Strategy
  \item Communication Media
  \item Delays
  \item Traveling Times
  \item Team Satisfaction
  \item Social Network Analysis
\end{itemize}

\subsection{Work Allocation Patterns}
\subsubsection{Separate by Subsystems}
\textbf{Indications:}
\begin{itemize}[itemsep=0pt]
  \item Modular product
  \item Clear and stable interfaces
\end{itemize}
\textbf{Pros:}
\begin{itemize}[itemsep=0pt]
  \item Low coupling, high cohesion
  \item Different development processes possible, only milestones needed
\end{itemize}
\textbf{Cons:}
\begin{itemize}[itemsep=0pt]
  \item More difficult integration
  \item Central components, cross-cutting features and non-functional requirements difficult to implement
\end{itemize}

\subsubsection{Separate by Process Steps}
\textbf{Indications:}
\begin{itemize}[itemsep=0pt]
  \item Limited technical resources
  \item Specific expertises
\end{itemize}
\textbf{Pros:}
\begin{itemize}[itemsep=0pt]
  \item Leverage time zones
  \item Gain experience with GSE
\end{itemize}
\textbf{Cons:}
\begin{itemize}[itemsep=0pt]
  \item High communication efforts across sites
  \item Inflexible to changing plans
\end{itemize}

\subsubsection{Separate by Releases}
\textbf{Indications:}
\begin{itemize}[itemsep=0pt]
  \item Previous releases remain stable
  \item Current release more critical than old ones
  \item Want to expose new site to product
\end{itemize}
\textbf{Pros:}
\begin{itemize}[itemsep=0pt]
  \item Current release still developed within one site
  \item Sites can learn about whole product
\end{itemize}
\textbf{Cons:}
\begin{itemize}[itemsep=0pt]
  \item Learning about whole product necessary
  \item Complicated to synchronize bug fixes
  \item Maintainers might feel disrespected
\end{itemize}

\subsubsection{Hybrid: Gradual Subsystem Split}
Combination of subsystem and release split.\\
\textbf{Indication:}
Interim step to take over responsibility for part of the product to deal with inexperience with the product. The long term goal would be to go to subsystem responsibilities.

\subsection{Collaboration Patterns}
\subsubsection{Training and Teambuilding}
\begin{description}[itemsep=0pt]
  \item[Tailored Training] Co-locate team members from multiple sites for training with standard (e.g. UML) and project-specific parts (e.g. simplified version of the target system)
  \item[Co-located Analysis Phase] Bring team members from multiple sites together to jointly review the high level architecture and requirements and develop functional specifications
\end{description}

\subsubsection{Selecting the Right Communication Media}
\begin{description}[itemsep=0pt]
  \item[Distributed Pair Programming] Virtual pair programming via application sharing
\end{description}

\subsubsection{Maintaining Cross-Site Relationship}
\begin{description}[itemsep=0pt]
  \item[Onsite Management Visits] Detailed status meetings with the project manager
  \item[Cross-Site Delegation] Delegate sent to another site to gain domain or project knowledge
  \item[Unfiltered Communication] Line Manager meets with developers from remote sites to help with problems
\end{description}

\subsubsection{Project Team Practices}
\paragraph{Establish and Prepare Team}
\begin{description}[itemsep=0pt]
  \item[Selected] Carefully select team members
  \item[Prepared] Train team members for distributed development
  \item[Culture Awareness] Being aware of cultural differences (intercultural training)
  \item[Team Mentor] Introduce a mentor to allow for on-the-job learning
\end{description}
\paragraph{Bring Team Members Together}
\begin{description}[itemsep=0pt]
  \item[Early Bonding] Focus on establish personal connections from early on
  \item[Short Engagements] Short-term assignments at other locations build personal connections and build up knowledge
  \item[Team Space] Local team rooms
\end{description}
\paragraph{Align Physical Team Meetings with Process}
\begin{description}[itemsep=0pt]
  \item[Together] Frequent team meetings (6-8 weeks for 10 days)
  \item[Iteration Connect] Synchronize length of iterations to frequency of team meetings
  \item[Completion United] Complete team meeting for project completion (for aprox. 3 weeks)
\end{description}
\paragraph{Virtual Communication}
\begin{description}[itemsep=0pt]
  \item[Smart Meetings] Agree on a time slot in the week where team members are available to contact
  \item[Team Connector] One team member on each side to manage communication
\end{description}
\paragraph{Project Identity and Communication Strategy}
\begin{description}[itemsep=0pt]
  \item[One Project] Same goal and priorities across sites
  \item[Communication Strategy] Create a communication plan: who to inform, stakeholders interests, communication means
  \item[Common Information Infrastructure] Set up global information structure
\end{description}
\paragraph{Development Process and Environment}
\begin{description}[itemsep=0pt]
  \item[Living Process] Establish shared development process, keep it up to date
  \item[Common Development Environment] One single development environment across all sites (24/7 support required)
\end{description}
\paragraph{Appreciate Team Commitment}
\begin{description}[itemsep=0pt]
  \item[Flexibility] Give team members flexibility outside common team time for additional efforts for communication
  \item[Full Credit] Equal credit for success for all team members
\end{description}

\newpage
