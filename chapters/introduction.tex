%!TEX root = ../report.tex

\section{Introduction}

Original pattern definition: A pattern is a three-part rule, which expresses a relation between a certain context, a problem, and a solution.

categorization of patterns:
\begin{itemize}
	\item Patterns for Development Activities
		\begin{itemize}
			\item Analysis Patterns
			\item Architecture Patterns
			\item Design Patterns
			\item Testing Patterns
		\end{itemize}
	\item Patterns for Crossfunctional Activities
		\begin{itemize}
			\item Process Patterns
			\item Agile Patterns
			\item Build and Release Management Patterns
		\end{itemize}
	\item Antipatterns (Smells)
\end{itemize}

\subsection{Conclusion}

\begin{itemize}
	\item  Patterns are Knowledge
	\item  Reusable source for solving problems
	\item  We acquire and describe knowledge to solve
	recurring design problems
	\item  Patterns are a great way to describe reusable
	knowledge
	\item  There are even Antipatterns: They are useful for
	describing lessons learned
	\item  Knowledge is often acquired by accidents or
	through failure
	\item  Learning from failures is important
	\item  Popper’s concept of falsification
\end{itemize}